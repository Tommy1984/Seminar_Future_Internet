\documentclass{semdoc}
% Template: $Id: t01_txt.tex,v 1.7 2000/05/23 12:13:37 bless Exp $
% -----------------------------------------------------------------------------
% Seminarbeitrag
% -----------------------------------------------------------------------------
% Kommentare beginnen mit einem %-Zeichen
\docbegin
% --> Oberhalb der Linie bitte nichts �ndern.
% ---------------------------------------------------------------------------
% \/ \/ \/ \/ \/ \/ \/ \/ \/ \/ \/ \/ \/ \/ \/ \/ \/ \/ \/ \/ \/ \/ \/ \/ \/ 
% Stellen, an denen etwas geaendert werden soll, sind wie hier gekennzeichnet.
% /\ /\ /\ /\ /\ /\ /\ /\ /\ /\ /\ /\ /\ /\ /\ /\ /\ /\ /\ /\ /\ /\ /\ /\ /\ 

%
% ---------------------------------------------------------------------------
% \/ \/ \/ \/ \/ \/ \/ \/ \/ \/ \/ \/ \/ \/ \/ \/ \/ \/ \/ \/ \/ \/ \/ \/ \/ 
% --> Bitte den Titel des Beitrages in die n�chste Zeile eintragen:
\title{Titel}
%
% --> ... und den Namen des Autors:
\author{Thomas Bersch}
% --> optional eine URL mailto:... oder http://www...
\authorURL{}
% /\ /\ /\ /\ /\ /\ /\ /\ /\ /\ /\ /\ /\ /\ /\ /\ /\ /\ /\ /\ /\ /\ /\ /\ /\ 
% -----------------------------------------------------------------------------
%
%
\maketitle
%
% ---------------------------------------------------------------------------
% \/ \/ \/ \/ \/ \/ \/ \/ \/ \/ \/ \/ \/ \/ \/ \/ \/ \/ \/ \/ \/ \/ \/ \/ \/ 
% --> ... und jetzt kommt die Zusammenfassung:
\begin{abstract}
%TODO: Abstract einf�gen
\end{abstract}
% /\ /\ /\ /\ /\ /\ /\ /\ /\ /\ /\ /\ /\ /\ /\ /\ /\ /\ /\ /\ /\ /\ /\ /\ /\ 
% -----------------------------------------------------------------------------
% ACHTUNG - ACHTUNG - ACHTUNG - ACHTUNG - ACHTUNG - ACHTUNG - ACHTUNG - ACHTUNG
% -----------------------------------------------------------------------------
% --> Im Text sollte \section der "h�chste" Gliederungsbefehl sein,
%     also bitte kein \chapter oder gar \part verwenden.
%     Der Text kann aber mit \section{}, \subsection{}, \subsubsection{}
%     untergliedert werden.
%
%     Bitte _neue_ Rechtschreibung verwenden!
%
%     Bitte nicht die Befehle \input oder \include verwenden, also den
%     Text _nicht_ in mehrere Dateien aufteilen! 
%
%     \newpage oder manuelle Zeilenumbr�che (\\) sollten ebenfalls nicht 
%     verwendet werden.
%
%     Bitte darauf achten, dass im _Quelltext_ ein Abschnitt nicht nur 
%     in einer Zeile steht (das macht z.B. Word beim Exportieren ohne
%     Zeilenumbruch), der Abschnitt sollte auch bei 80 Zeichen pro Zeile
%     noch lesbar sein, d.h. Zeilenumbr�che im Quelltext bitte entsprechend
%     einf�gen.
%
%     Referenzen auf andere Abschnitte sind bitte mit \ref{...}, wie 
%     anschliessend gezeigt, anzugeben und nicht etwa als Text wie
%     "siehe auch Abschnitt 2.2"
%
%     Anf�hrungszeichen sind nicht einfach so "Text" einzugeben, sondern so:
%     "`Text"', andernfalls gibt es Fehler.
%
%     Das Zeichen ~ erzeugt einen Leerraum an dem aber nicht getrennt wird.
%     Weitere Trennstellen:
%     "- = zus�tzliche Trennstelle
%     "| = Vermeidung von Ligaturen und m�gliche Trennung (bsp: Schaf"|fell)
%     "~ = Bindestrich an dem keine Trennung erlaubt ist (bsp: bergauf und "~ab)
%     "= = Bindestrich bei dem Worte vor und dahinter getrennt werden d�rfen
%     "" = Trennstelle ohne Erzeugung eines Trennstrichs (bsp: und/""oder)
%
%     Weiterer Hinweis: 
%     KEIN Glossar erstellen! Abk�rzungen sind im Text zu erkl�ren!
%     Der Text ist mit z.B. mit ispell auf Schreibfehler zu pr�fen 
%     (am besten aus dem Emacs heraus, 
%     Men�: Edit -> Spell -> Select Deutsch8, dann Edit -> Spell -> Check Buffer)
%
% ---------------------------------------------------------------------------
% \/ \/ \/ \/ \/ \/ \/ \/ \/ \/ \/ \/ \/ \/ \/ \/ \/ \/ \/ \/ \/ \/ \/ \/ \/ 

\section{Einleitung} % in die Klammern die Ueberschrift 
\label{tD2_Einleitung} % und ein Label

Hier steht die Einleitung!

\section{Grundlagen} % in die Klammern die Ueberschrift 
\label{tD2_Grundlagen} % und ein Label

\section{Verwandte Arbeiten}
\label{tD2_Verwandte_Arbeiten}

\section{Implementierung}
\label{tD2_Implementierung}

\section{Evaluierung}
\label{tD2_Evaluierung}

\section{Zusammenfassung}
\label{tD2_Zusammenfassung}

% /\ /\ /\ /\ /\ /\ /\ /\ /\ /\ /\ /\ /\ /\ /\ /\ /\ /\ /\ /\ /\ /\ /\ /\ /\ 
% -----------------------------------------------------------------------------
% Bitte keinen Text mehr unterhalb dieser Zeile eintragen!!
% -----------------------------------------------------------------------------
%
% --> �blicherweise wird nur die Literatur aufgelistet, die auch referenziert
%     wird. M�chte man auch nichtreferenzierte Literatur einschlie�en, so
%     kann man dies mit \nocite{<citelabel>} tun.
\nocite{*}
%     In die folgende Zeile sollte die ben�tigte Literaturdatenbankdatei
%     eingetragen werden (im Normalfall nicht zu �ndern):
\bibliography{t01_txt}
%
\docend
%%% end of document
