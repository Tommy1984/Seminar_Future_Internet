\documentclass{semdoc}
% Template: $Id: t01_txt.tex,v 1.7 2000/05/23 12:13:37 bless Exp $
% -----------------------------------------------------------------------------
% Seminarbeitrag
% -----------------------------------------------------------------------------
% Kommentare beginnen mit einem %-Zeichen
\docbegin
% --> Oberhalb der Linie bitte nichts ndern.
% ---------------------------------------------------------------------------
% \/ \/ \/ \/ \/ \/ \/ \/ \/ \/ \/ \/ \/ \/ \/ \/ \/ \/ \/ \/ \/ \/ \/ \/ \/ 
% Stellen, an denen etwas geaendert werden soll, sind wie hier gekennzeichnet.
% /\ /\ /\ /\ /\ /\ /\ /\ /\ /\ /\ /\ /\ /\ /\ /\ /\ /\ /\ /\ /\ /\ /\ /\ /\ 

%
% ---------------------------------------------------------------------------
% \/ \/ \/ \/ \/ \/ \/ \/ \/ \/ \/ \/ \/ \/ \/ \/ \/ \/ \/ \/ \/ \/ \/ \/ \/ 
% --> Bitte den Titel des Beitrages in die nchste Zeile eintragen:
\title{Seminararbeit - R3: Resilient Routing Reconfiguration}
%
% --> ... und den Namen des Autors:
\author{Thomas Bersch}
% --> optional eine URL mailto:... oder http://www...
\authorURL{}
% /\ /\ /\ /\ /\ /\ /\ /\ /\ /\ /\ /\ /\ /\ /\ /\ /\ /\ /\ /\ /\ /\ /\ /\ /\ 
% -----------------------------------------------------------------------------
%
%
\maketitle
%
% ---------------------------------------------------------------------------
% \/ \/ \/ \/ \/ \/ \/ \/ \/ \/ \/ \/ \/ \/ \/ \/ \/ \/ \/ \/ \/ \/ \/ \/ \/ 
% --> ... und jetzt kommt die Zusammenfassung:
\begin{abstract}
%TODO: Abstract einfügen
Hier steht der Abstract
\end{abstract}
% /\ /\ /\ /\ /\ /\ /\ /\ /\ /\ /\ /\ /\ /\ /\ /\ /\ /\ /\ /\ /\ /\ /\ /\ /\ 
% -----------------------------------------------------------------------------
% ACHTUNG - ACHTUNG - ACHTUNG - ACHTUNG - ACHTUNG - ACHTUNG - ACHTUNG - ACHTUNG
% -----------------------------------------------------------------------------
% --> Im Text sollte \section der "hchste" Gliederungsbefehl sein,
%     also bitte kein \chapter oder gar \part verwenden.
%     Der Text kann aber mit \section{}, \subsection{}, \subsubsection{}
%     untergliedert werden.
%
%     Bitte _neue_ Rechtschreibung verwenden!
%
%     Bitte nicht die Befehle \input oder \include verwenden, also den
%     Text _nicht_ in mehrere Dateien aufteilen! 
%
%     \newpage oder manuelle Zeilenumbrche (\\) sollten ebenfalls nicht 
%     verwendet werden.
%
%     Bitte darauf achten, dass im _Quelltext_ ein Abschnitt nicht nur 
%     in einer Zeile steht (das macht z.B. Word beim Exportieren ohne
%     Zeilenumbruch), der Abschnitt sollte auch bei 80 Zeichen pro Zeile
%     noch lesbar sein, d.h. Zeilenumbrche im Quelltext bitte entsprechend
%     einfgen.
%
%     Referenzen auf andere Abschnitte sind bitte mit \ref{...}, wie 
%     anschliessend gezeigt, anzugeben und nicht etwa als Text wie
%     "siehe auch Abschnitt 2.2"
%
%     Anfhrungszeichen sind nicht einfach so "Text" einzugeben, sondern so:
%     "`Text"', andernfalls gibt es Fehler.
%
%     Das Zeichen ~ erzeugt einen Leerraum an dem aber nicht getrennt wird.
%     Weitere Trennstellen:
%     "- = zustzliche Trennstelle
%     "| = Vermeidung von Ligaturen und mgliche Trennung (bsp: Schaf"|fell)
%     "~ = Bindestrich an dem keine Trennung erlaubt ist (bsp: bergauf und "~ab)
%     "= = Bindestrich bei dem Worte vor und dahinter getrennt werden drfen
%     "" = Trennstelle ohne Erzeugung eines Trennstrichs (bsp: und/""oder)
%
%     Weiterer Hinweis: 
%     KEIN Glossar erstellen! Abkrzungen sind im Text zu erklren!
%     Der Text ist mit z.B. mit ispell auf Schreibfehler zu prfen 
%     (am besten aus dem Emacs heraus, 
%     Men: Edit -> Spell -> Select Deutsch8, dann Edit -> Spell -> Check Buffer)
%
% ---------------------------------------------------------------------------
% \/ \/ \/ \/ \/ \/ \/ \/ \/ \/ \/ \/ \/ \/ \/ \/ \/ \/ \/ \/ \/ \/ \/ \/ \/ 

\section{Einleitung}
\label{tD2_Einleitung}
Heutige Netze wie beispielsweise das  IP-basierte Internet, 
bestehen in der Regel aus einer Vielzahl einzelner Netzknoten, 
die nicht notwendigerweise direkt, 
sondern �ber mehrere Knoten hinweg miteinander verbunden sind.
Sollen Information zu einem entfernten Knoten transportiert werden, 
m�ssen diese von den dazwischen liegenden Netzknoten weitergereicht werden, 
bis sie den Zielknoten erreicht haben.
Die Wahl der zu verwendeten Zwischensysteme, 
also eines Weges wird als Routing bezeichnet.
Ein wichtiger Aspekt dabei ist die Optimierung des Weges bez�glich einer Metrik, 
wie z.B die anfallenden Kosten, der k�rzeste Weg, oder die aktuelle Netzauslastung.

\section{Grundlagen}
\label{tD2_Grundlagen}
%TODO Routing (nun detailierter)


%TODO Label Switching
Label Switching
%TODO MPLS
Multi-Protokoll-Label-Switching (MPLS) 

%TODO flow-based routing representation

%TODO Network resiliency
%TODO Deutscher Begriff f�r "Network Resiliency"
In \cite{WWMA10} wird 'Network resiliency' als die F�higkeit eines Computernetzes definiert, 
sich schnell und problemlos von einer Reihen von Fehlern oder Unterbrechungen zu erholen. 
Eine zunehmend wichtiger werdende Eigenschaft f�r moderne IP-basierte Netze.

%TODO Notationen

%TODO Verwandte Arbeiten
Im Gegensatz zu \cite{WWMA10}, der Grundlage dieser Seminararbeit, 
behandeln bissherige Arbeiten entweder Routing im Fehlerfall oder Routing f�r variable Netzauslastungen.



\section{Resilient Routing Reconfiguration}

%TODO: Bessere Begriffe f�r Base-Routing und Protection Routing suchen
R3 - Resilient Routing Reconfiguration besteht prinzipiell aus zwei Teilen, 
einer vom eignetlichen Netzbetrieb unabh�ngigen Vorberechnung (Offline-Phase) 
sowohl eines Base-Routings als auch eines Protection-Routings 
und einer st�ndigen Rekonfiguration (Online-Phase) des Protection Routings w�hrend des Netzbetriebs.

\subsection{Vorberechnung (Offline-Phase)}
In der Vorberechnungsphase wird, sofern noch nicht gegeben, 
ein Routing $r$ f�r eine Verkehrsmatrix $d$ berechnet, 
Auf dieser Grundlage wird ein Protection-Routing $p$ berechnet, 

\subsection{Rekonfiguration (Online-Phase)}

\subsection{Beweise}

\subsection{Erweiterungen}
%TODO �berschrifften besser formulieren
\subsubsection{Behandlung von unterschiedlichem Verkehrsaufkommen}
\subsubsection{Behandlung von realisitschen Fehlerszenarien}
\subsubsection{Behanldung von Priorisiertem Verkehr}
\subsubsection{Abw�gung zwischne Performance unter normalen Bedinngungen und im Fehlerfall}
\subsubsection{Abw�gung zwischne Netzauslastung und Verz�gerungszeiten} 

\section{Implementierung}
%Wie scheibt man das?
%Die haben das so implementiert? oder Das kann man so implementieren
%Ist die Implementierung �berhaupt wichtig
%TODO MPLS-ff


\section{Evaluierung}

\subsection{Simulation}

\subsection{Reale Implementierung}


\section{Zusammenfassung}





% /\ /\ /\ /\ /\ /\ /\ /\ /\ /\ /\ /\ /\ /\ /\ /\ /\ /\ /\ /\ /\ /\ /\ /\ /\ 
% -----------------------------------------------------------------------------
% Bitte keinen Text mehr unterhalb dieser Zeile eintragen!!
% -----------------------------------------------------------------------------
%
% --> blicherweise wird nur die Literatur aufgelistet, die auch referenziert
%     wird. Mchte man auch nichtreferenzierte Literatur einschlieen, so
%     kann man dies mit \nocite{<citelabel>} tun.
\nocite{*}
%     In die folgende Zeile sollte die bentigte Literaturdatenbankdatei
%     eingetragen werden (im Normalfall nicht zu ndern):
\bibliography{tD2_txt}
%
\docend
%%% end of document
